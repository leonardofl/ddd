\documentclass[a4paper, 12pt]{article}
\usepackage[utf8]{inputenc}
\usepackage[brazil]{babel}
\usepackage[top=3cm,left=3cm,right=2cm,bottom=2cm]{geometry} % para as margens
\usepackage{graphicx} % para as figuras  
\usepackage{alltt}                   

\title{Resumo do livro ``Domain-Driven Design''}
\author{Leonardo Leite}

\begin{document}

\maketitle

\textbf{Atenção: } este documento trata-se ainda de uma versão em desenvolvimento.

\section{Introdução}

Este documento é um resumo do livro ``Domain-Driven Design, Atacando as Complexidades no Coração do Software'', escrito por Eric Evans em 2003. Os objetivos deste resumo são:

\begin{itemize}
\item Em primeiro lugar, fazer um exercício de fixação e assimilação de minha leitura do livro.
\item Servir de referência rápida para que eu tente aplicar os conceitos do DDD no desenvolvimento de software.
\item Fornecer a outras pessoas uma ideia rápida sobre o que trata o livro Domain-Driven Design, assim como apresentar alguns de seus principais conceitos.
\end{itemize}

Uma consideração importante ao leitor: embora minha prática do desenvolvimento de software já tenha vários pontos de acordo com o DDD, este resumo foi escrito logo após minha leitura do livro. Ou seja, não se trata de um texto baseado em minha experiência com DDD, mas somente na leitura do livro.

Destaco também que o livro que eu li foi a primeira edição da tradução para o português (2009). Aliás, uma tradução não tão boa. O pior exemplo de tradução que achei foi a tradução de \emph{graphs} (grafos) e \emph{edges} (arestas) respectivamente por gráficos e bordas. Não sei dizer se a tradução em si foi melhorada na segunda edição da tradução (2011).

Minha impressão geral sobre o livro: gostei bastante da mentalidade do autor, bem condizente com princípios da Programação Extrema (XP) e ao mesmo tempo com uma boa dose de pragmatismo. No entanto o livro é um tanto quanto grande e verboso. Embora seja definitivamente um clássico, já é relativamente antigo (2003). Dessa forma, acho que talvez possam existir por aí livros mais resumidos que possam transmitir a essência do DDD sem tantas considerações e mais direto ao ponto considerando o atual estado-da-arte do desenvolvimento de software. Não que o livro tenha deixado de ser atual, mas algumas considerações menores poderiam hoje ser deixadas de lado. Ao mesmo tempo, já surgiram alguns paradigmas modernosos como \emph{micro-serviços}, que são bem relacionados com o tema do livro. Portanto, sua decisão de ler ou não o livro vai depender bastante do quanto você deseja se aprofundar no assunto.

Outra consideração bem interessante é, pra mim, foi bem positivo ter demorado pra ler esse livro. Já tinha ouvido falar do livro há alguns anos, mas somente agora finalmente o priorizei. E isso foi bom porque a experiência recente que eu tive desenvolvendo um sistema com um complicado modelo de domínio me ajudou bastante a ler o livro com melhor proveito. Outros sistemas em que trabalhara antes não tinham um modelo de domínio assim tão complexo. Ler o livro e comparar o que o autor descreve com suas próprias experiências é de grande valor para uma melhor leitura do livro. Dessa forma, dou a ousada sugestão de que se você ainda é um desenvolvedor iniciante leia apenas um livro básico (ou mesmo esse resumo) sobre o assunto e espere alguns anos até ler o livro Domain-Driven Design.

Bom, chega de enrolação. Vamos ao que interessa...

\section{A linguagem onipresente}

\section{Camadas de um software}

\section{Padrões da camada do domínio}



\end{document}
